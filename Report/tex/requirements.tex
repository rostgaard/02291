\chapter{Requirements}
\noindent

\section{Domain Analysis}
%From PD. Keep in mind while writing.
%Create a glossary explaining the notions and terminologies discovered when analysing the domain of the toll system. Create a class diagram of the domain. Describe the basic workflows / business proceses of the toll system as activity diagrams.
\subsection{Glossary}
\subsubsection{Station-related terminology}
\begin{description}

  \item[Entry:] A point where the Customer enters the Motorway, consists of a selection of Toll Lanes and,Toll Station and Toll Both.
  \item[Exit:] A point where the Customer leaves the Motorway, consists of a selection of Toll Lanes and a Toll Station.
  \item[Express Toll Lane:] Toll Lane prepared to work with Toll Tags
  \item[Motorway:] The road for the Customers consisting of at least one Entry and Exit
  \item[Normal Toll Lane:] Toll Lane prepared to work with Single Tickets
  \item[Toll Lane] a path on the Entry on which the Customer (one at the time) can enter the Motorway. It has a Computer.
  \item[Toll Station:] A place where the Station Manager resides and overseers the Toll Lanes
  \item[Toll System:] The infrastructure for managing the Motorway, used by Cashiers, Managers and Customers, also called Enterprise.


\end{description}
\subsubsection{Payment \& Ticket terminology}
\begin{description}
  \item[Cash] one of the payment methods on the Normal Toll Lane for a Single Ticket, requires a Cashier using a Cash Register. The availability of this method on the Toll Lane is confirmed by the Indicator
  \item[Check In] Operation of entering the Motorway on the Entry performed by the Customer with use of Single Ticket or Toll Tag. Successful Check In yields the Toll Lane Barrier to open
  \item[Check Out] Operation of leaving the Motorway on the Exit performed by the Customer with use of Single Ticket or Toll Tag. Successful Check Out yields the Toll Lane Barrier to open
  \item[Credit card] One of the payment methods on the Normal Toll Lane for a Single Ticket, does not require a Cashier, but is performed through self service with a Credit Card Reader.
  \item[Single Ticket] One of the payment methods available for Motorway Customers, has a fixed price based on the Vehicle type and the Ticket Validity of 24 hours after being printed
  \item[Ticket Validity] State of the Single Ticket stating, if it can be used for a successful Check Out operation. Voids after 24 hours of sale or after a Check Out
  \item[Toll Charge] The price for the single travel performed by the Customer who uses the Toll Tag, its value depends on the Travel Distance and the type of the Vehicle
  \item[Toll Tag:] A wireless beacon placed in the Vehicle which allows the Customer driving it to use the Express Toll Lanes
  \item[Travel Distance:] Distance between the chosen Entry and Exit, where the corresponding Check in and Check out were performed by the Customer using the Toll Tag
\end{description}

\subsubsection{Actors}
\begin{description}

  \item[Bank:] entity responsible for processing payments for Toll Tags and Credit Card, reached through a dedicated communication channel.
  \item[Cashier:] - one of the users of the Toll System, he is responsible for either selling Single Tickets with Cash on the Toll Lanes with Cash Register or resolving problems with Ticket Validity at the Exit
  \item[Customer:] A Vehicle owner, one of the Toll System users
  \item[Manager] - Toll System user responsible for managing Toll Station (Toll Station Manager) or the whole Toll System (Enterprise Manager). Can generate Reports and change Single Ticket or Toll Tag price

\end{description}

\subsubsection{Hardware components}
\begin{description}

  \item[Antenna] - Part of the Express Lanes, connects to the Toll Tag to perform a Check In or Check out
  \item[Barrier] - part of the Toll Lane, stops the Vehicle from passing through it unless a valid Check In or Check Out is performed
  \item[Cash register] - a device placed in the Normal Tall Lane, operated by Cashier
  \item[Client] - A Computer used by Managers for registration and generating reports - registration of Managers
  \item[Computer] - Client for Station, Client and server for Toll System or simple terminal for Toll Lane to be used by the Toll System users.
  \item[Credit card reader] - a device placed in every Normal Tall Lane, operated by Customer
  \item[Internet] - a channel for communication between the Toll System, Bank and Customers
  \item[Printer] - a device placed in every Toll Lane and Stations, used for printing Single Tickets and Reports
  \item[Server] - connects all computers in Toll Station (Station Server) or Toll System (Enterprise Server). Server is managed by a Client (Station or Enterprise).
  \item[Single ticket reader] - part of the Normal Toll Lanes at the Exits, checks the Validity of Single Tickets presented by the Customers during the Check Out operation.
  \item[Toll lane indicator] - indicates, if the Toll Lane allows Cash payment, i. e. if the Cash Register is installed and the Cashier is present
  \item[Touch screen] - interface for the Computer
  \item[Vehicle] - car, motorbike or truck operated by the Customer, can have a Toll Tag attached to it
  \item[Web server] - Toll System’s gate to the Internet
\end{description}
\subsubsection{Other}
\begin{description}
  \item [Customer Notification] Information sent to all Customers via Internet, initialized by Enterprise Manager
  \item[Report] Summary of the Toll System performance, generated by the Manager with use of the Client
  \item[Subscription] Agreement between a toll tag customer and the toll enterprise to enable the toll enterprise to make recurring withdrawals based on toll tag usage.
\end{description}

\section{Actors}
Actors and their capabilities.
\begin{itemize}
\item Cashier

\begin{itemize}
  \item Launch lane
  \item Sell ticket / check in
  \item Solve check out problems
\end{itemize}
\end{itemize}

\begin{itemize}
\item Station manager

\begin{itemize}
  \item Generate station reports
\end{itemize}
\end{itemize}

\begin{itemize}
\item Enterprise manager

\begin{itemize}
  \item Generate enterprise reports
  \item Change toll rates
  \item Send notifications to customers
\end{itemize}
\end{itemize}

\begin{itemize}
\item Customer

\begin{itemize}
  \item Check-in
  \item Check-out
  \item Buy toll tag
\end{itemize}
\end{itemize}

%TODO: <SHORT TEXT>
%TODO: <ACTIVITY DIAGRAM>

\begin{figure}

\caption{Activity Diagram of the Basic Workflow}
\end{figure}

\section{Use cases} \martin \pawel \\
\subsection{Customer point of view}
\begin{usecase}
\addtitle{Use Case 1}{Check-in on a normal, manned lane} 
\addfield{Author:}{\martin}
\addfield{Description:}{Customer goes through a check-in procedure buying a single ticket on a lane with a cashier}
\addfield{Primary Actor:}{Customer}
\addfield{Notes:}{Payment can be done with cash or credit card}

\additemizedfield{Stakeholders}{
	\item Customer
	\item Cashier
}
\additemizedfield{Preconditions:}{
      \item Customer has arrived at toll station from outside the toll system
      \item Lane is manned by a cashier 
      \item Barrier is down
}
\additemizedfield{Postconditions:}{
      \item Barrier is lifted and the customer is now on the toll road
}
\addscenario{Main Success Scenario:}{
      \item Looks for cashier lane and drives up to it
      \item Pays the required amount of money
      \item Takes ticket
      \item Drives through the barrier
}
\addscenario{Extensions:}{
	\item[1.a] Does not have enough money to pay
		\begin{enumerate}
		\item[1.] Being guided off and drives away
		\end{enumerate}
}
\end{usecase}
\begin{usecase}
\addtitle{Use Case 2}{Check-in on a normal, unmanned lane} 
\addfield{Author:}{\martin}
\addfield{Description:}{Customer goes through a check-in procedure buying a single ticket on a lane without a cashier}
\addfield{Primary Actor:}{Customer}
\addfield{Notes:}{Payment can only be done only with credit card}

\additemizedfield{Stakeholders}{
	\item Customer
}
\additemizedfield{Preconditions:}{
      \item Customer has arrived at toll station from outside the toll system
      \item Barrier is down
}
\additemizedfield{Postconditions:}{
      \item Barrier is lifted and the customer is now on the toll road
}
\addscenario{Main Success Scenario:}{
      \item Looks for unmanned normal lane and drives up to it
      \item Picks vehicle type and is presented with amount to be paid
      \item Inserts credit card into credit card reader and enters credit card PIN
      \item Takes ticket
      \item Drives through the barrier
}
\addscenario{Extensions:}{
	\item[3.a] Credit card PIN is declined
		\begin{enumerate}
		\item[1.] Retries by returning to 2
		\end{enumerate}
	\item[3.b] Credit card is declined entirely
		\begin{enumerate}
		\item[1.] Being guided off and drives away
		\end{enumerate}
}
\end{usecase}
\begin{usecase}
\addtitle{Use Case 3}{Check-in on an express lane} 
\addfield{Author:}{\martin}
\addfield{Description:}{Customer goes through a check-in procedure using a toll tag on an express lane}
\addfield{Primary Actor:}{Customer}

\additemizedfield{Stakeholders}{
	\item Customer
}
\additemizedfield{Preconditions:}{
      \item Customer has arrived at toll station from outside the toll system
      \item Barrier is down
}
\additemizedfield{Postconditions:}{
      \item Barrier is lifted and the customer is now on the toll road
}
\addscenario{Main Success Scenario:}{
      \item Looks for express lane and drives up to it
      \item Drives at specified speed and waits for toll tag validation to complete
      \item Drives through the barrier
}
\addscenario{Extensions:}{
	\item[2.a] Toll tag not validated
		\begin{enumerate}
		\item[1.] Retries
		\item[2.] If still unsuccessful, drives to a normal lane to check-in there
		\end{enumerate}
}
\end{usecase}
\begin{usecase}
\addtitle{Use Case 4}{Check-out with a single ticket} 
\addfield{Author:}{\martin}
\addfield{Description:}{Customer goes through a check-out procedure using a single ticket}
\addfield{Primary Actor:}{Customer}

\additemizedfield{Stakeholders}{
	\item Customer
}
\additemizedfield{Preconditions:}{
      \item Customer has arrived at toll station from the toll road
      \item Barrier is down
}
\additemizedfield{Postconditions:}{
      \item Barrier is lifted and the customer has now left the toll road system
}
\addscenario{Main Success Scenario:}{
      \item Looks for normal lane and drives up to it
      \item Inserts ticket to ticket reader
      \item Drives through the barrier
}
\addscenario{Extensions:}{
	\item[2.a] Ticket not validated
		\begin{enumerate}
		\item[1.] Waits for cashier to show up
		\item[2.] Follows cashiers directions
		\end{enumerate}
}
\end{usecase}
\begin{usecase}
\addtitle{Use Case 5}{Check-out with a toll tag} 
\addfield{Author:}{\martin}
\addfield{Description:}{Customer goes through a check-out procedure using a toll tag}
\addfield{Primary Actor:}{Customer}

\additemizedfield{Stakeholders}{
	\item Customer
}
\additemizedfield{Preconditions:}{
      \item Customer has arrived at toll station from the toll road
      \item Barrier is down
}
\additemizedfield{Postconditions:}{
      \item The customer has now left the toll road system
}
\addscenario{Main Success Scenario:}{
      \item Looks for toll tag lane and drives up to it
      \item Drives at specified speed and waits for toll tag validation to complete
      \item Drives through the barrier
}
\addscenario{Extensions:}{
	\item[2.a] Toll tag not validated
		\begin{enumerate}
		\item[1.] Waits for cashier to show up
		\item[2.] Follows cashiers directions
		\end{enumerate}
}
\end{usecase}
\begin{usecase}
\addtitle{Use Case 6}{Toll tag purchase} 
\addfield{Author:}{\pawel}
\addfield{Description:}{Customer procures a toll tag for him/herself}
\addfield{Primary Actor:}{Customer}

\additemizedfield{Stakeholders}{
	\item Customer
	\item Toll enterprise
}
\additemizedfield{Postconditions:}{
      \item The customer gets the toll tag by mail and installs it in the vehicle
}
\addscenario{Main Success Scenario:}{
      \item Visits the toll enterprise website
      \item A form is displayed
      \item Fills a form with vehicle, personal and bank information
      \item Submits the form online
}
\addscenario{Extensions:}{
	\item[1.a] Arrives at a toll station
		\begin{enumerate}
		\item[1.] Takes a form
		\item[2.] Fills a form with vehicle, personal and bank information
		\item[3.] Submits the form in person
		\end{enumerate}
}
\end{usecase}
\subsection{Cashier point of view}
\begin{usecase}
\addtitle{Use Case 7}{Launching a toll lane} 
\addfield{Author:}{\pawel}
\addfield{Description:}{A cashier launches a normal, manned toll lane}
\addfield{Primary Actor:}{Cashier}
\addfield{Notes:}{A "normal" lane is not an express lane}
\additemizedfield{Stakeholders}{
	\item Cashier
}
\additemizedfield{Preconditions:}{
      \item The lane is a normal lane
}
\additemizedfield{Postconditions:}{
      \item The toll lane is launched
}
\addscenario{Main Success Scenario:}{
      \item Enters a toll lane
      \item Identifies as cashier using a touch screen
}
\end{usecase}
\begin{usecase}
\addtitle{Use Case 8}{Checking a vehicle in} 
\addfield{Author:}{\pawel}
\addfield{Description:}{A cashier checks a vehicle in}
\addfield{Primary Actor:}{Cashier}
\addfield{Notes:}{A "normal" lane is not an express lane. Payment can be done with both cash and credit card.}
\additemizedfield{Stakeholders}{
	\item Cashier
	\item Customer
}
\additemizedfield{Preconditions:}{
      \item The lane is a normal lane
      \item The lane has been launched
      \item A vehicle has arrived at the lane
}
\additemizedfield{Postconditions:}{
      \item Barrier is lifted and the vehicle drives through
}
\addscenario{Main Success Scenario:}{
      \item Check the vehicle type
      \item Informs the customer about the price
      \item Receives the payment and processes it
      \item Requests ticket to be printed
      \item Gives ticket to the customer
}
\addscenario{Extensions:}{
	\item[3.a] Customer is unable to pay the specified amount
		\begin{enumerate}
		\item[1.] No ticket printed and the vehicle is not allowed to pass through
		\end{enumerate}
}
\end{usecase}
\begin{usecase}
\addtitle{Use Case 9}{Checking a vehicle out} 
\addfield{Author:}{\pawel}
\addfield{Description:}{A cashier checks a vehicle out manually in case of malfunction of the automatic system}
\addfield{Primary Actor:}{Cashier}
\addfield{Notes:}{An example problem can be a customer without validated ticket arriving to the lane or a toll tag malfunctioning. The use case abstracts from the issues that may arise and how they ought to be solved - suffice to say, that it is in the cashiers discretion and/or according to some internal rules and guidelines.}
\additemizedfield{Stakeholders}{
	\item Cashier
	\item Customer
}
\additemizedfield{Preconditions:}{
      \item A problem with the check-out has arisen
}
\additemizedfield{Postconditions:}{
      \item Barrier is lifted and the vehicle drives through
}
\addscenario{Main Success Scenario:}{
      \item Identifies the problem
      \item Solves the problem
}
\end{usecase}
\subsection{Station manager point of view}
\begin{usecase}
\addtitle{Use Case 10}{Identification} 
\addfield{Author:}{\pawel}
\addfield{Description:}{A station manager logs into the station client}
\addfield{Primary Actor:}{Station manager}
\addfield{Notes:}{The identification procedure is abstracted from}
\additemizedfield{Stakeholders}{
	\item Station manager
}
\additemizedfield{Postconditions:}{
      \item The station manager is logged into the station client
}
\addscenario{Main Success Scenario:}{
      \item Identifies as manager by pressing a button
}
\end{usecase}
\begin{usecase}
\addtitle{Use Case 11}{Station report generation} 
\addfield{Author:}{\pawel}
\addfield{Description:}{A station manager generates a periodical station statistics report}
\addfield{Primary Actor:}{Station manager}
\addfield{Notes:}{An invalid time period is one where the start date/time comes after the end date/time.}
\additemizedfield{Stakeholders}{
	\item Station manager
}
\additemizedfield{Preconditions:}{
      \item The station manager is logged into the station system client (see use case identification).
}
\additemizedfield{Postconditions:}{
      \item The report is generated. The report includes tallies for check-ins/outs by vehicle type and statistics for usage of single tickets and toll tags.
}
\addscenario{Main Success Scenario:}{
      \item Enters a desired time period for the report
      \item Request report to be generated
}
\addscenario{Extensions:}{
	\item[1.a] Invalid time period is entered
		\begin{enumerate}
		\item[1.] The time period has to be entered again
		\end{enumerate}
}
\end{usecase}
\subsection{Enterprise manager point of view}
\begin{usecase}
\addtitle{Use Case 12}{Identification} 
\addfield{Author:}{\pawel}
\addfield{Description:}{An enterprise manager logs into the enterprise client}
\addfield{Primary Actor:}{Enterprise manager}
\addfield{Notes:}{The identification procedure is abstracted from}
\additemizedfield{Stakeholders}{
	\item Enterprise manager
}
\additemizedfield{Postconditions:}{
      \item The enterprise manager is logged into the enterprise client
}
\addscenario{Main Success Scenario:}{
      \item Identifies as manager by pressing a button
}
\end{usecase}
\begin{usecase}
\addtitle{Use Case 13}{Toll rate change} 
\addfield{Author:}{\pawel}
\addfield{Description:}{An enterprise manager changes the toll rates}
\addfield{Primary Actor:}{Enterprise manager}
\addfield{Notes:}{The rate for toll tags is a fixed price per kilometer. The rate for single tickets is a fixed price for a trip}
\additemizedfield{Stakeholders}{
	\item Enterprise manager
}
\additemizedfield{Preconditions:}{
      \item The enterprise manager is logged into the station system client (see use case identification).
}
\additemizedfield{Postconditions:}{
      \item The toll rates are consistent for all stations
}
\addscenario{Main Success Scenario:}{
      \item Selects new rate for toll tags and/or single tickets
      \item Toll rates are updated for all stations 
}
\addscenario{Extensions:}{
	\item[1.a] Invalid rate is entered
		\begin{enumerate}
		\item[1.] The rate has to be entered again
		\end{enumerate}
	\item[2.a] A rate update on one (or more) station(s) has failed
		\begin{enumerate}
		\item[1.] The rates remain unchanged for the whole system
		\end{enumerate}
}
\end{usecase}
\begin{usecase}
\addtitle{Use Case 14}{Notifying customers about changes} 
\addfield{Author:}{\pawel}
\addfield{Description:}{An enterprise manager sends out a notification to the customers}
\addfield{Primary Actor:}{Enterprise manager}
\addfield{Notes:}{The notification method can be either by traditional mail or email}
\additemizedfield{Stakeholders}{
	\item Enterprise manager
	\item Customers
}
\additemizedfield{Preconditions:}{
      \item The enterprise manager is logged into the station system client (see use case identification).
}
\additemizedfield{Postconditions:}{
      \item The notifications are sent out to customers
}
\addscenario{Main Success Scenario:}{
      \item Enters notification text
      \item Selects notification method
}
\end{usecase}
\begin{usecase}
\addtitle{Use Case 15}{Enterprise report generation} 
\addfield{Author:}{\pawel}
\addfield{Description:}{An enterprise manager generates a periodical enterprise statistics report}
\addfield{Primary Actor:}{Enterprise manager}
\addfield{Notes:}{An invalid time period is one where the start date/time comes after the end date/time.}
\additemizedfield{Stakeholders}{
	\item Enterprise manager
}
\additemizedfield{Preconditions:}{
      \item The enterprise manager is logged into the enterprise system client (see use case identification).
}
\additemizedfield{Postconditions:}{
      \item The report is generated. The report includes tallies for check-ins/outs by vehicle type and statistics for usage of single tickets and toll tags for all stations.
}
\addscenario{Main Success Scenario:}{
      \item Enters a desired time period for the report
      \item Request report to be generated
}
\addscenario{Extensions:}{
	\item[1.a] Invalid time period is entered
		\begin{enumerate}
		\item[1.] The time period has to be entered again
		\end{enumerate}
}
\end{usecase}

\section{CRC cards} \anna \piotr \\
\begin{figure}[H]
\includegraphics[scale=0.35]{\imgdir "CRC1"}
\centering
\caption{CRC cards}
\label{fig:crc_1}
\end{figure}

\begin{figure}[H]
\includegraphics[scale=0.35]{\imgdir "CRC2"}
\centering
\caption{CRC cards}
\label{fig:crc_2}
\end{figure}

\section{Functional Requirements} \pawel \\
%From PD. Keep in mind while writing.
%Based on the basic workflows and on the problem description, identify the functional requirements using use cases. Create a use case diagram showing all identified use cases. Out of the use cases identified and shown in the use case diagram, select 4–6 use cases (cf. Sect. 4) according to their priority for the customer (who is the owner of the toll system). The selected use cases form the basis of the remaining sections, e.g. the design and the validation sections. For the selected use cases create detailed use case descriptions according to the following template: 
%1. Use Case Name
%2. Summary
%3. Actors
%4. Preconditions
%5. Basic course of events
%6. Alternative paths
%7. Postconditions
%8. Notes
%9. Author
% You may use the usecase.sty for easy formatting.
The functional requirements describe what the system is supposed to do. Below is a list of such requirements applicable to the toll system:
\begin{description}
  \item[RQ1] The system shall be able to enforce toll collection on motorways.
  \item[RQ2] A toll station shall be placed at every entry and exit of a motorway.
  \item[RQ3] It shall be possible to check vehicles in and out on toll lanes located at every toll station.
  \item[RQ4] There shall be two types of toll lanes: express and normal.
  \item[RQ5] The toll lane type shall be displayed on an indicator.
  \item[RQ6] Two toll collection methods shall be available: single tickets and toll tags.
  \item[RQ7] The toll collection on an express lane shall be performed automatically and wirelessly using toll tags.
  \item[RQ8]The toll collection on a normal lane shall be performed manually by selling single tickets.

  \item[RQ9]It shall be possible to pay for a single ticket using a credit card or cash.
  \item[RQ10]A single ticket shall be valid for one trip only and for a maximum duration of 24 hours.

  \item[RQ11] It shall be possible to purchase toll tags at every toll station or on the Internet.
  \item[RQ12] The price for purchasing a toll tag shall be fixed.
  \item[RQ13] A toll tag shall be bound to one vehicle only.
  \item[RQ14] A toll tag shall be bound to a bank account.
  \item[RQ15] All toll fees incurred on a toll tag shall be charged monthly from the bank account it is bound to.

  \item[RQ16] Integration of all devices and tasks responsible for a check in/out shall be performed by a toll lane computer.
  \item[RQ17] Information from every toll lane computer shall be forwarded to its toll station server and stored there.
  \item[RQ18] Information from every toll station server shall be forwarded to an enterprise server and stored there.

  \item[RQ19] It shall be possible to generate periodical statistical reports with information about a single toll stations drift.
  \item[RQ20] The reports shall include information about types of vehicles checked in and frequency of single ticket and toll tag usage.
  \item[RQ21] It shall be possible to generate reports with same information as above for all toll stations simultaneously.

  \item[RQ22] It shall be possible to change toll rates.
  \item[RQ23] The toll rates shall depend on the vehicle types.
  \item[RQ24] Single tickets shall have a fixed price for a trip.
  \item[RQ25] Toll tags shall have a fixed rate per kilometer.
  \item[RQ26] It shall be possible to send arbitrary notifications to customers via standard mail or email.

\end{description}
\section{Non-Functional Requirements} \trevon \pawel \\
The non-functional requirements describe how the system is supposed to be. Below is a list of such requirements applicable to the toll system:

\begin{description}
  \item[NRQ1] In order to minimise the queues on the normal lanes without a cashier, the system's user interface for the customers shall be easy to use. A customer arriving at a normal lane and paying with a credit card shall be able to do so without assistance within no more than 1 minute. 
  \item[NRQ2] The processes of ticket purchase and validation shall take no longer than 1 minute each.
  \item[NRQ3] It shall be possible to distinguish the lane type from a distance of at least 100 meters.
  \item[NRQ4] The fault rate of the wireless toll collection (with antennas and toll tags) shall be at most 1 vehicle per day per toll station.
  \item[NRQ5] Station and enterprise generation shall be fast. It shall take no longer than 30 seconds to generate and display a report.
\end{description}


\begin{figure}
\includegraphics[scale=0.4]{\imgdir "Domain Model"}
\caption{Domain model}
\label{fig:domain_model}
\end{figure}
