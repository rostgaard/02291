\chapter{Requirements}
\noindent

\section{Domain Analysis}
%From PD. Keep in mind while writing.
%Create a glossary explaining the notions and terminologies discovered when analysing the domain of the toll system. Create a class diagram of the domain. Describe the basic workflows / business proceses of the toll system as activity diagrams.

\section{Functional Requirements}
%From PD. Keep in mind while writing.
%Based on the basic workflows and on the problem description, identify the functional requirements using use cases. Create a use case diagram showing all identified use cases. Out of the use cases identified and shown in the use case diagram, select 4–6 use cases (cf. Sect. 4) according to their priority for the customer (who is the owner of the toll system). The selected use cases form the basis of the remaining sections, e.g. the design and the validation sections. For the selected use cases create detailed use case descriptions according to the following template: 
%1. Use Case Name
%2. Summary
%3. Actors
%4. Preconditions
%5. Basic course of events
%6. Alternative paths
%7. Postconditions
%8. Notes
%9. Author
% You may use the usecase.sty for easy formatting.

\section{Non-Functional Requirements}