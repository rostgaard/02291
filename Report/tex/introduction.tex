\chapter{Introduction}
\trevon \\\\
%From PD. Keep in mind while writing.
%The introduction should describe the steps actually used to solve the modelling task and why these stepswere chosen. For example, were techniques from agile modelling used (which?), were CRC cards used,domain-driven design, explorative modelling, etc.?
\noindent
The purpose of this document is to model a system for collecting toll on highways. The central part of the system are toll stations, which are placed at highway entrances and exits. Each station has several types of lanes, which are used by the customers to check in and out of the highway. Customers can do so by purchasing single tickets, which are valid for a single journey or by using toll tags, The single tickets are purchased upon arrival at a toll station with cash or credit card. The toll tags can be ordered online or on the toll stations. The toll tags enable the customers to expedite their entering and leaving the highways, since they are operated wirelessly and automatically, whereas single tickets require some manual actions to be performed. \\ The entire system comprises of many toll stations, each with hardware and personnel required to perform the basic operation of the system - checking vehicles in and out and collecting toll. Meanwhile, it is also possible to query statistical data from the system in form of reports, which give overview of the single stations or the entire enterprise. Finally, some administrative tasks like changing the toll rates and sending notifications to customers are also included in the system.

\section{Methodology Used}
The workflow management technique used throughout this project was Kanban. The use cases/features were listed as work items and each item was processed as follows:

\begin{itemize}
  \item Requirements 
  \item Development
  \item Test
  \item Implement
\end{itemize}

In the requirements phase one use case is fully analysed of its requirements before it is advanced to the development phase. The requirements were categorized into two groups: functional and non-functional requirements. The functional requirements for the toll system were clear from the system specification document, however, several discussions were needed to resolve ambiguities.

The development phase utilised agile modelling technique SCRUM, which enabled us to model the system incrementally and iteratively. While modelling the system, due to changes in understanding on how the system should behave, using SCRUM was a great choice.  The modelling tasks were divided among smaller groups of pairs, which not only helped to complete the tasks faster, but also meant that people were constantly checking each others work and ensuring common understanding of arising issues.

CRC cards were developed in order for conceptual modelling of the system. The use of CRC cards gives the developers an overview of what each class is responsible for and what it does in the system. They provided a base on which the component and class diagrams were created later.

The testing of the model was done by using acceptance test. These were expressed with FIT tables. Furthermore, use case realisation was validated by creating sequence diagrams for each use case. By doing so, greater understanding of the system was achieved and it made much easier to spot errors. Upon spotting errors or unmet requirements, we went back in the process and amended the issues. By testing the model we managed to validate each if each use case meets the requirement of the system.

Implementation of the system model was out of scope for this project and therefore implementation/platform specific considerations are not part of this report.
\\

\textbf{Note about images in the report:} The images have been scaled down in order to fit the pages. They can, however, be readily zoomed into without quality loss.