\chapter{Introduction}
%From PD. Keep in mind while writing.
%The introduction should describe the steps actually used to solve the modelling task and why these stepswere chosen. For example, were techniques from agile modelling used (which?), were CRC cards used,domain-driven design, explorative modelling, etc.?
\noindent
This document contains the modelling of a toll system which could be used on any highway road. The system needs to implemented in toll stations which are placed at the entrance and exit point of the highway. Customers can use the highway by purchasing a single ticket which is at fixed price at the entrance or by using a toll tag which could be bought via the system online. The customer have the option to purchase the ticket by cash or by credit card at the toll stations. In order to identify the toll tag there are wireless sensors which are installed on the toll station to read the toll tag on a vehicle. Once the toll tag is read by the sensor, the system will authenticate the vehicle, then the barrier at the toll station will open allowing the user to use the highway. The authentication of the vehicle takes place at both entrance and the exit toll stations. Station managers can query statistical data from the system which can give the manager an overview of the system usage. Another additional feature of the system is that it enables enterprise manager to send email or mail notification to customers about prices changes.
\section{Methodology Used}

The software development workflow technique we used was Kanban, where the use cases/feature were listed as work items and each item was processed through the following process:

\begin{itemize}
  \item Requiremnts 
  \item Development
  \item Test
  \item Implement
\end{itemize}


In the requiremnts phase one use case is fully analyzed of its requirements before its is advanced to the development phase. The requirements were categorized into two, Functional and Nonfunctional Requirements, were functional requirements for the toll system were clear from the system specification document, however there were few ambiguous areas. 

The software system development phase was modeled using agile modeling technique scrum which enabled us to model the system incrementally and iteratively. While modeling the system, due to changes on how the system should behave due to the ambiguity of the requirements using scrum was a great choice.  The modeling tasks were divided among smaller groups of pairs which helped to complete the tasks faster but also there was another person to judge the other. 

CRC cards were developed in order for conceptual modelling of the system.  The use of CRC cards will give the developer an overview of what each class is responsible for and what it does in the system. 

The testing of the software system model was done by using acceptance test, we achieved this by using language to express acceptance test called Fit. By testing the model we managed to validate each if each use case meets the requirement of the system. After the validation tests, we managed to find some use cases where the requirement was not met so we have to go back and change the development.  Using Fit we have designed Fit tables which shows all the test cases done for the system, this can be seen in chapter 3.

Implementation of the system model was not part of the requirement of the project, therefore we did not consider implementation of the system.

