\chapter{Design}
%This section contains the design of the system. It is only necessary to create a design realizing the use cases identified in Sect. 4. Explain any non-trivial design decision that you make, and state any assumptions that you make. 2
\noindent

\section{Component Design}
%Define the components of the system with their ports (i.e., a port is a set of required and provided   interfaces). For each provided interface of a port provide the protocol state machine. For hardware components that one would buy off the shelf, like antenna, ticket reader, barrier, . . . , it is sufficient to provide the ports with their protocol state machines without going into the details of their possible implementation.

\section{Class Design}
%Define the classes used to implement the components. Some components can be implemented as a single classes; other components can be implemented by a set of classes. Provide any necessary class invariants using OCL constraints. Specify the contracts of non-trivial operations using OCL constraints (pre:/post:). Describe each class, their interaction, and the contract of operations also using informal text. Note that for hardware components that one would buy off the shelf, like antenna, ticket reader, barrier, ... , it is not necessary to detail their implementation (cf. Sect. 4).

\section{Behaviour Design}
%Describe the behaviour of each non-trivial class using object life cycle state machines.