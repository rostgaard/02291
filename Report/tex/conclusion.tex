\chapter{Conclusion}
\pawel \kim \\
During the course of this project, a system for toll collection on roads have been designed and modelled. To begin with, the domain model and a basic activity workflow of the system has been modelled. Based on that and requirement elicitation from the project description document, a number of detailed use cases has been written. Based on the use cases, CRC cards have been produced and used to model component and detailed class diagrams. Additionally, for each use case, acceptance tests have been defined using fit tables. Finally, use case realisation has been validated using sequence diagrams. \\
The validation step (acceptance tests and sequence diagrams) has been extremely useful for finding errors and making adjustments to the models. It has provided numerous insights into how the system might actually be realised and thereby vastly increased the understanding of the system as a whole and its components separately. \\
The produced documentation, albeit not complete, should provide sufficient information for skilled developers to implement the system.

\section{Work distribution}
Where appropriate, names of mainly responsible people are indicated throughout the report. In case of no indication, all group members collaborated on that part of the report. \\ All in all, the workload was distributed equally amongst group members.

\section{Experience with the project}
%This section should cotain the experiences with the project. For example, what was learned, what are the things you can improve next time, and what you did make good this time. (This will not be graded!)
Taking this course has been, roughly equivalent, of banging my little toe against a large -- and very solid -- credenza. The pain is excruciating, and the one question that runs through my mind is ``why?!''. -- Grade that!
