\chapter{Conclusion}
\pawel \\\\
During the course of this project, a system for toll collection on roads have been designed and modelled. To begin with, the domain model and a basic activity workflow of the system has been modelled. Based on that and requirement elicitation from the project description document, a number of detailed use cases has been written. Based on the use cases, CRC cards have been produced and used to model component and detailed class diagrams. Additionally, for each use case, acceptance tests have been defined using fit tables. Finally, use case realisation has been validated using sequence diagrams. \\
The validation step (acceptance tests and sequence diagrams) has been extremely useful for finding errors and making adjustments to the models. It has provided numerous insights into how the system might actually be realised and thereby vastly increased the understanding of the system as a whole and its components separately. \\
The produced documentation, albeit not complete, should provide sufficient information for skilled developers to implement the system.

\section{Work distribution}
The workload was distributed equally amongst group members. The initial domain and workflow analysis, as well as requirement elicitation and creation of glossary has been done as a collaborative effort by all group members. Subsequently, the work was performed pairwise on use case basis. The 15 use cases have been divided equally among 3 pairs of group members. Subsequently, each pair has produced all diagrams and documentation relevant to their use case. The report was written and proofread by all members as a collaborative effort. The names of people with main responsibility for given chapters are stated in the beginning of each chapter. The use cases were divided as follows: \\
\begin{center}
\begin{tabular}{ c || c }
  Pair & Use Cases \\ \hline \hline
  \martin \\ \pawel & 1, 2, 3, 8, 13 \\ \hline
  \anna \\ \piotr & 6, 7, 10, 11, 12 \\ \hline
  \kim \\ \trevon & 4, 5, 9, 14, 15 \\ 
\end{tabular}
\end{center}

\section{Experience with the project}
%This section should cotain the experiences with the project. For example, what was learned, what are the things you can improve next time, and what you did make good this time. (This will not be graded!)
%Doing this project has been, roughly equivalent, of banging my little toe against a large -- and very solid -- credenza. The pain is excruciating, and the one question that runs through my mind is ``why?!''. This project ought to be defenstrated. -- Grade that!
The project has provided us with valuable experience in the whole process of modelling a system from a mere short description document. It has given us insight into eliciting the requirements from such a document and modelling the system around them - making sure that they are all fulfilled and that none are "made up". Furthermore, we gained a deep understanding of many different UML models and other methods like FIT tables and CRC cards; most importantly however, it is now clear to us how they relate to and complement each other. \\
All of the aforementioned have come together to create a complete adventure of writing a thorough document, on which an implementation of the system, albeit mostly trivial, could be based. \\
We were able to follow an effective and efficient development process revolving around use cases and agile methodologies. Working on the project has therefore been very satisfying.